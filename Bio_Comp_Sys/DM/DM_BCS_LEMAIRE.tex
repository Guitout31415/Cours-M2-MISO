\documentclass[12pt,a4paper]{article}
\usepackage[utf8]{inputenc}
\usepackage[T1]{fontenc}
\usepackage{amsmath}
\usepackage{graphicx}
\usepackage{amssymb}
\newcommand{\R}{\mathbb{R}}
%\usepackage[top=2cm,bottom=2cm,left=1cm,right=1cm]{geometry}

\title{DM BCS Partie 1}
\author{LEMAIRE Guillaume M2 MISO}
\date{Février 2023}

\begin{document}

\maketitle

\section*{Exercice 1}
\subsection*{Q1)}

La loi de conservation est : $$\forall t\in\R_+; P(t)+2P_2(t)+3P_3(t)=P(0)+2P_2(0)+3P_3(0)$$
Cette formule à été trouvée grâce au package python $\mathsf{sympy}$.

\subsection*{Q2)}

Il est plausible que 2 réactions aient les mêmes constantes de réactions si elles ont des mécanismes similaires, ou sont effectuées dans des conditions similaires (pression, température, pH, ...)

\subsection*{Q3)}

\begin{itemize}
\item On a $2$ réactions réversibles dans 2 classes, donc le réseau est faiblement réversible.

\item $\delta = 4-2-\text{rang}
\begin{pmatrix}
-2 & 2 & -1 & 1 \\
1 & -1 & -1 & 1 \\
0 & 0 & 1 & -1 \\
\end{pmatrix} \\= 4-2-\text{rang}
\begin{pmatrix}
-2 & 0 & -1 & 0 \\
1 & 0 & -1 & 0 \\
0 & 0 & 1 & 0 \\ 
\end{pmatrix} \\= 4-2-2 = 0$
\end{itemize}
\noindent\newline
Comme le système est faiblement réversible et que $\delta=0$, on applique le théorème de déficience zéro.
Il existe donc un unique équilibre pour chaque classe de comptabilité.

\newpage
\subsection*{Q4)}
\begin{figure}[h!]
	\includegraphics[scale=0.6]{"graph_ex1_q4.png"}
\end{figure}
On remarque que :
\begin{itemize}
\item Si $P_3(0) > $
\item Si $P(0) \leq P_2(0)$, alors $P$ et $P_2$ sont décroissantes.
\item Si $P(0) > P_2(0)$, alors $P$ est décroissante et $P$.
\end{itemize}

\subsection*{Q5)}

On peut conjecturer que $p_{3,0}$ augmente si T augmente. 

En effet, plus la quantité totale de protéines augmente, il y aura plus de protéines disponibles pour réagir, et donc de produire plus de $P_3$ à l'équilibre.

\newpage
\subsection*{Q6)}
\begin{figure}[h!]
	\includegraphics[scale=0.6]{"graph_ex1_q6.png"}
\end{figure}

La courbe est strictement croissante divergente et convexe.

\subsection*{Q7)}

%%%%%%%%%%%%%%%%%%%%%%%%%%%%%%%%%%%%%%%%%%%

\section*{Exercice 2}
\subsection*{Q1)}

\subsection*{Q2)}

\subsection*{Q3)}

\subsection*{Q4)}

\subsection*{Q5)}

\subsection*{Q6)}

\subsection*{Q7)}

\end{document}
